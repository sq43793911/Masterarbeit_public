	\pagestyle{fancy}
	\section{Zusammenfassung und Ausblick} \label{sec:Zusammenfassung}
	In der vorliegenden Arbeit wird sowohl die numerische als auch die experimentelle Modalanalyse eines Werkzeugschaftes beim HSC-Fräsen beschrieben. Dabei werden die Auswirkungen von verschiedenen Parametern auf die Eigenfrequenzen des Schaftes analysiert. Wegen der hohen Rotationsgeschwindigkeit des Werkzeughalters wurde die simulative Modalanalyse durch eine nichtlinearen Finite-Elemente-Methode, in Kombination mit dem Prinzip von Hamilton, durchgeführt und mit \Matlab programmiert. Aufgrund des experimentellen Versuchsaufbaus mit dem kabelgebundenen Anregungshammer und dem Beschleunigungssensor ist es nicht möglich, Versuche unter Rotation durchzuführen. Deshalb wurde am oberen Ende des Werkzeugschafts eine Radialkraft eingeleitet, um die Wirkung der Rotation in Verbindung mit einer Exzentrizität nachzubilden.\\
	
	Zunächst werden die notwendigen theoretischen Grundlagen erklärt, um das Berechnungsmodell zur numerischen Simulation umzusetzen. Im Speziellen wird hierfür die nichtlineare Finite-Elemente-Methode verwendet. Anschließend werden die Eigenfrequenzen und Deformationen des Schaftes für verschiedene Parameter simuliert. Bei den Experimenten werden die niedrigsten Eigenfrequenzen in Abhängigkeit der Vorspannkraft (entspricht einem Exzentrizitätsfehler $e$ bei gleichzeitiger Rotation) ermittelt.\\
	
	Die Ergebnisse werden anschließend in einem separaten Kapitel ausgewertet. Beispielsweise zeigen die Simulationen eine Aufspaltung der Eigenfrequenzen durch die Rotation. Weiterhin ist eine schwache Abhängigkeit der Eigenfrequenzen von der Vorspannkraft bei den experimentellen Ergebnisse zu sehen. So wurde eine tendenzielle Versteifung für die Eigenfrequenzen des ersten Modes mit zunehmender Vorspannkraft festgestellt.\\
	
	Schließlich ist festzustellen, dass die ermittelten Erkenntnisse zur Konstruktion von schwingungsarmen Werkzeugschäften und deren Optimierung verwendet werden können. Für das weitere Vorgehen wird empfohlen, mehrere Messungen durchzuführen und die Ergebnisse anschließend zu mitteln. Damit können Messungenauigkeiten durch zufällige Störungen minimiert werden.