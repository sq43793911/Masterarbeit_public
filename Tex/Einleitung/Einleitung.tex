	\pagestyle{fancy}
	\section{Einleitung}\label{sec:einleitung}
	Eine wichtige Fertigungstechnologie in der Metallverarbeitung ist das Hochgeschwindigkeitszerspanen (englisch \textit{High Speed Cutting}, HSC). Dabei handelt es sich um ein fortschrittliches Zerspanungsverfahren, welches hohe Effizienz, Qualität und geringen Verbrauch kombiniert. Eine Reihe von Problemen, die bei herkömmlichen Zerspanungsverfahren auftreten, wurden durch die Anwendung von HSC reduziert. Im Vergleich zum herkömmlichen Zerspanen werden die Schnittgeschwindigkeit und die Vorschubgeschwindigkeit um mehrere Stufen erhöht.\\
	
	Mit zunehmender Schnittgeschwindigkeit nimmt die Abtragsrate pro Zeiteinheit zu, die Schnittzeit nimmt ab und die Verarbeitungseffizienz steigt. Dadurch wird der Herstellungszyklus des Produkts verkürzt und die Wettbewerbsfähigkeit auf dem Markt verbessert. Gleichzeitig verringert eine geringe Schnittmenge mit schneller Bewegungen den Werkzeugverschleiß. Gleichzeitig werden die Schnittkraft und die thermische Spannungsverformung des Werkstücks verringert. Das führt auf bessere Verarbeitungsmöglichkeiten von Teilen und Materialien mit geringer Steifigkeit sowie dünnwandige Teile. Aufgrund der Verringerung der Schnittkraft und der Erhöhung der Drehzahl ist die Arbeitsfrequenz des Schneidsystems weit von der Eigenfrequenz niedriger Ordnung der Werkzeugmaschine entfernt, wodurch die Oberflächenrauheit verringert wird. Die Oberflächenrauheit des Werkstücks ist am empfindlichsten gegenüber der Frequenz niedriger Ordnung. Die HSC-Technologie kann in Anwendungsgebieten eingesetzt werden, die eine hohe Anforderungen an Zerspanleistung und eine hohe Oberflächenqualität erfordern, also insbesondere in der Werkzeugbearbeitung und der Formenbearbeitung. Zum Beispiel gibt es komplexe dreidimensionale Formen, welche höchste Maßgenauigkeit und Oberflächengenauigkeit aufweisen müssen \cite{huseynov2015entwicklung}. \\
	
	Wegen der Frequenzempfindlichkeit des Zerspanungsverfahrens sind die Eigenfrequenzen des Werkzeugs wichtige Parameter. Deshalb wird die Modalanalyse verwendet, um die dynamischen Eigenschaften von Systemen im Frequenzbereich zu untersuchen. Die Eigenfrequenzen sind im konstruktiven Ingenieurbau sehr wichtig, da es unerlässlich ist, dass diese nicht mit den erwarteten Erregungsfrequenzen übereinstimmen (Resonanz). Falls es trotzdem zur Resonanz kommt, kann das Objekts strukturellen Schaden erleiden. Aus methodologischer Sicht ist die Modalanalyse in numerische und experimentelle Modalanalyse zu unterteilen. In dieser Arbeit wird die numerische Modalanalyse mit der Finite-Elemente-Methode (FEM) und die experimentelle Modalanalyse mit Experimenten betrachtet.\\
	
	Es soll die Auswirkung von Exzentrizitätsfehlern auf die Eigenfrequenzen des Werkzeugschaftes beim HSC-Fräsen durch simulative und experimentelle Modalanalyse analysiert werden. Aufgrund der hohen Geschwindigkeiten des Werkzeughalters wird eine nichtlineare Finite-Elemente-Methode, in Kombination mit dem Prinzip von Hamilton, für die Simulation verwendet. Anschließend wird die FE Simulation mit \Matlab durchgeführt. Im experimentellen Teil wird die sogenannte Anregungshammermethode für die experimentellen Modalanalyse benutzt. Außerdem wird das CAD Programm \SolidWorks zur Darstellung der Bauteilen benutzt, die im Experiment angewendet werden. Schließlich werden die Ergebnisse von Simulation und Experiment verglichen und analysiert.
	