	\pagestyle{fancy}
	\fancyhead[RO,LE]{Anhang A: Programme}
	\setcounter{page}{1} 
	\fancyhead[LO,RE]{A-\thepage}
	\addcontentsline{toc}{paragraph}{Anhang A: Programme}
	\section*{Anhang A: Programme}
	\subsection*{Hauptprogramm}
	\begin{lstlisting}
	clear
	close all
	%###########################################
	% main Program  Ver-7.3     08.06.2020  Qian Sun  
	%###########################################
	%% parameter
	E=2.1e11;         % N/m^2
	D0=0.01;           % Durchmesser m
	R=D0/2;
	d=0.0005;          % Wandstaeker m
	r=R-d;
	A=pi*(R^2-r^2);   % Flaeche m^2
	l=0.33;           % m
	rho=7800;         % Dichte in [kg/m^3]
	V0=A*l*rho;       % Volume [m^3]
	mu=rho*A;         % Massenbelegung in [kg/m]
	Nel=20;           % number of elements in bend
	Nno=Nel+1;        % number of nodes in bend in linear_ansatz
	% Nno=Nel*2+1;        % number of nodes in qudra-ansatz
	le=l/Nel;         % length of an element
	I=pi*(R^4-r^4)/4;  % Flaechentraegheitsmoment
	It=2*I;     % Torsionstraegheitsmoment
	v=0.28;                % Poissonzahl
	G=E/(2*(1+v));        % Schubmodul
	c=0;                  % Feder-Steifigkeit [N/m]
	m=0.0303;                % Masse  [kg]
	
	Omega=40;    % Drehgeschwindigkeit [rad/s]
	e=-0;      % Exzentrizitaet      [m]
	q=6;          % Freiheitsgrad
	
	Fx=50;                      % force [N]
	Fy=0;                      % force [N]
	Fz=0;                      % force [N]
	M=0;                       % moment [N*m]
	
	FVec= zeros(q*Nel,1);       % empty global force Vektor 
	FVec(end-5)=Fx;
	FVec(end-4)=Fy;
	FVec(end-2)=Fz;
	FVec(end)=M;
	
	uMat=[];
	vMat=[];
	vxMat=[];
	wMat=[];
	wxMat=[];
	
	%% define empty matrice
	Kt=zeros(Nno*q);  % empty global stiffnes-matrix 
	M=zeros(Nno*q);   % empty global mass-matrix 
	B=zeros(Nno*q);   
	Q=zeros(Nno*q);
	
	
	Ae=zeros(2*q,q*Nno,Nel);
	for ie=1:Nel
	for i=1:2*q
	Ae(i,q*(ie-1)+i,ie)=1;
	end
	end
	
	u=zeros(Nno,1);
	w=zeros(Nno,1);
	v=zeros(Nno,1);
	wx=zeros(Nno,1);
	vx=zeros(Nno,1);
	phi=zeros(Nno,1);
	
	Nloop=5;
	Ux=0;
	Vx=0;
	Wx=0;
	
	%% main program
	for j=1:Nloop
	
	if j==1
	
	for k=1:Nel                                      % loop over every element
	[Kte,Me,Be,Qe,Ce,MeM,KteM,BeM,QeM] = Elementroutine_n_linear(A,E,rho,le,Ux,Vx,Wx,I,It,G,Omega,e,c,m);
	if k==Nel
	Kt=Kt+Ae(:,:,k)'*Kte*Ae(:,:,k)+Ae(:,:,k)'*Ce*Ae(:,:,k)+Ae(:,:,k)'*KteM*Ae(:,:,k);
	M =M+Ae(:,:,k)'*Me*Ae(:,:,k)+Ae(:,:,k)'*MeM*Ae(:,:,k);              
	B =B+Ae(:,:,k)'*Be*Ae(:,:,k)+Ae(:,:,k)'*BeM*Ae(:,:,k);
	Q =Q+Ae(:,:,k)'*Qe*Ae(:,:,k)+Ae(:,:,k)'*QeM*Ae(:,:,k);
	else
	Kt=Kt+Ae(:,:,k)'*Kte*Ae(:,:,k);             % place the distribution of every element to right place in global stiffnes-matrix
	M =M+Ae(:,:,k)'*Me*Ae(:,:,k);               % place the distribution of every element to right place in global mass-matrix
	B =B+Ae(:,:,k)'*Be*Ae(:,:,k);
	Q =Q+Ae(:,:,k)'*Qe*Ae(:,:,k);
	end
	end
	
	
	else
	
	for k=1:Nel                                     % loop over every element
	
	%                                  Ux=Fx/E/A;
	%                                  Vx=(vx(k+1)+vx(k))/2;
	%                                  Wx=(wx(k+1)+wx(k))/2;
	%            
	Ux=(u(k+1)-u(k))/le;
	Wx=(w(k+1)-w(k))/le;
	Vx=(v(k+1)-v(k))/le;
	[Kte,Me,Be,Qe,Ce,MeM,KteM,BeM,QeM] = Elementroutine_n_linear(A,E,rho,le,Ux,Vx,Wx,I,It,G,Omega,e,c,m);
	if k==Nel
	Kt=Kt+Ae(:,:,k)'*Kte*Ae(:,:,k)+Ae(:,:,k)'*Ce*Ae(:,:,k)+Ae(:,:,k)'*KteM*Ae(:,:,k);
	M =M+Ae(:,:,k)'*Me*Ae(:,:,k)+Ae(:,:,k)'*MeM*Ae(:,:,k);               % place the distribution of every element to right place in global mass-matrix
	B =B+Ae(:,:,k)'*Be*Ae(:,:,k)+Ae(:,:,k)'*BeM*Ae(:,:,k);
	Q =Q+Ae(:,:,k)'*Qe*Ae(:,:,k)+Ae(:,:,k)'*QeM*Ae(:,:,k);
	else
	Kt=Kt+Ae(:,:,k)'*Kte*Ae(:,:,k);             % place the distribution of every element to right place in global stiffnes-matrix
	M =M+Ae(:,:,k)'*Me*Ae(:,:,k);               % place the distribution of every element to right place in global mass-matrix
	B =B+Ae(:,:,k)'*Be*Ae(:,:,k);
	Q =Q+Ae(:,:,k)'*Qe*Ae(:,:,k);
	end
	end
	end
	
	for zz=1:q
	Kt(1,:) = [];
	Kt(:,1) = [];
	M(1,:) = [];
	M(:,1) = [];
	B(1,:) = [];
	B(:,1) = [];
	end
	
	QVec=diag(Q);
	QVec(1:q)=[];
	
	AVec=FVec+QVec;
	
	P=Kt\AVec;
	for yy=1:q
	P=[0;P];
	end
	
	for xx=1:Nno
	n=(xx-1)*q+1;
	u(xx)=P(n);
	w(xx)=P(n+1);
	wx(xx)=P(n+2);
	v(xx)=P(n+3);
	vx(xx)=P(n+4);
	phi(xx)=P(n+5);
	end
	
	if j==Nloop
	
	else
	Kt= zeros(q*Nno);                               % empty global stiffnes-matrix 
	M = zeros(q*Nno);                                % empty global mass-matrix 
	B = zeros(q*Nno);   
	Q = zeros(q*Nno);
	end
	
	% uMat=[uMat,u];
	% vMat=[vMat,v];
	% vxMat=[vxMat,vx];
	% wMat=[wMat,w];
	% wxMat=[wxMat,wx];
	
	end
	
	%%
	% define system-matrix
	null=zeros(size(M));
	Eins = eye(size(M));
	SysMat=[null,Eins; -inv(M)*Kt,-inv(M)*B];
	
	[V,D]=eig(SysMat);
	
	temp_d = diag(D);
	org_d = temp_d;
	for i=1:Nel*6
	temp_d(i,:)=[];
	end
	
	[nd, sortindex] = sort(temp_d);
	temp_v = V(:,sortindex);
	temp_f = -imag(nd)/(2*pi);
	
	%%
	lVec = zeros (Nno, 1);     % vector with node coordinates
	
	for k = 1: Nno
	lVec(k)=l/Nel*(k-1);
	end
	
	
	%% plot
	
	figure('Name','Programm')
	grid on
	
	subplot(1,3,1)
	plot(lVec,u);
	title('u(x)')
	subplot(1,3,2)
	plot(lVec,w);
	title('w(x)')
	subplot(1,3,3)
	plot(lVec,v);
	title('v(x)')
	\end{lstlisting}
	
	\subsubsection*{Elementroutine}
	\begin{lstlisting}
	%###########################################
	% Elementroutine  Ver-7.3     08.06.2020  Qian Sun
	%###########################################
	function [Kte,Me,Be,Qe,Ce,MeM,KteM,BeM,QeM] = Elementroutine_n_linear(A,E,rho,le,Ux,Vx,Wx,I,It,G,Omega,e,c,m)
	% Elementroutine: compute Kte, Me
	
	%%
	% define empty Matrix
	Kteux=zeros(12);
	Ktev=zeros(12);
	Ktevx=zeros(12); 
	Ktevxx=zeros(12);
	Ktew=zeros(12);
	Ktewx=zeros(12); 
	Ktewxx=zeros(12);
	Ktephi=zeros(12);
	Ktephix=zeros(12);
	
	Me=zeros(12);
	
	Be=zeros(12);
	Qe=zeros(12,1);
	
	
	%%
	xiVec=[-sqrt(3/7+2/7*sqrt(6/5)),-sqrt(3/7-2/7*sqrt(6/5)),sqrt(3/7-2/7*sqrt(6/5)),sqrt(3/7+2/7*sqrt(6/5))]; % define sampling points for Gauss-quadrature  
	wVec=[(18-sqrt(30))/36,(18+sqrt(30))/36,(18+sqrt(30))/36,(18-sqrt(30))/36];    % weights for sampling points of Gauss-quadrature 
	%%
	
	for i=1:length(xiVec)
	xi=xiVec(i);
	w =wVec(i);
	
	% define N, Nx, Nxx vector linear 
	N=[0.5-xi/2, 0, 0, 0, 0, 0, 0.5+xi/2, 0, 0, 0, 0, 0; ...
	0, 1/2-(3*xi)/4+(xi^3)/4, 1/4-xi/4-(xi^2)/4+(xi^3)/4, 0, 0, 0, 0, 1/2+(3*xi)/4-(xi^3)/4, -1/4-xi/4+(xi^2)/4+(xi^3)/4, 0, 0, 0; ...
	0, -3/4+(3*xi^2)/4, -1/4-xi/2+(3*xi^2)/4, 0, 0, 0, 0, 3/4-(3*xi^2)/4, -1/4+xi/2+(3*xi^2)/4, 0, 0, 0; ...
	0, 0, 0, 1/2-(3*xi)/4+(xi^3)/4, 1/4-xi/4-(xi^2)/4+(xi^3)/4, 0, 0, 0, 0, 1/2+(3*xi)/4-(xi^3)/4, -1/4-xi/4+(xi^2)/4+(xi^3)/4, 0; ...
	0, 0, 0, -3/4+(3*xi^2)/4, -1/4-xi/2+(3*xi^2)/4, 0, 0, 0, 0, 3/4-(3*xi^2)/4, -1/4+xi/2+(3*xi^2)/4, 0; ...
	0, 0, 0, 0,  0, 0.5-xi/2, 0, 0,  0,  0,  0, 0.5+xi/2];
	
	Nx=[-0.5, 0, 0,0, 0, 0, 0.5, 0, 0, 0, 0, 0; ...
	0, -3/4+(3*xi^2)/4, -1/4-xi/2+(3*xi^2)/4, 0, 0, 0,   0, 3/4-(3*xi^2)/4, -1/4+xi/2+(3*xi^2)/4, 0, 0, 0; ...
	0, 3*xi/2, -1/2+3*xi/2, 0, 0, 0, 0, -3*xi/2, 1/2+3*xi/2, 0, 0, 0; ...
	0, 0, 0, -3/4+(3*xi^2)/4,  -1/4-xi/2+(3*xi^2)/4, 0, 0, 0, 0, 3/4-(3*xi^2)/4, -1/4+xi/2+(3*xi^2)/4, 0; ...
	0, 0, 0, 3*xi/2, -1/2+3*xi/2, 0, 0, 0, 0, -3*xi/2, 1/2+3*xi/2, 0; ...
	0, 0, 0, 0, 0, -0.5, 0, 0, 0, 0, 0, 0.5]*(2/le);
	
	Nxx=[0, 0, 0, 0, 0, 0, 0, 0, 0, 0, 0, 0; ...
	0, 3*xi/2, -1/2+3*xi/2, 0, 0, 0, 0, -3*xi/2, 1/2+3*xi/2, 0, 0, 0; ...
	0, 3/2, 3/2, 0, 0, 0, 0, -3/2, 3/2, 0, 0, 0; ...
	0, 0, 0, 3*xi/2, -1/2+3*xi/2, 0, 0, 0, 0, -3*xi/2, 1/2+3*xi/2, 0; ...
	0, 0, 0, 3/2, 3/2, 0, 0, 0, 0, -3/2, 3/2, 0; ...
	0, 0, 0, 0, 0, 0, 0, 0, 0, 0, 0, 0]*((2/le)^2);
	
	Nu=N(1,:);
	Nw=N(2,:);
	Nv=N(4,:);
	Nphi=N(6,:);
	
	Nux=Nx(1,:);
	Nwx=Nx(2,:);
	Nvx=Nx(4,:);
	Nphix=Nx(6,:);
	
	Nuxx=Nxx(1,:);
	Nwxx=Nxx(2,:);
	Nvxx=Nxx(4,:);
	Nphixx=Nxx(6,:);
	
	% compute Kte, Me, Be, Qe for sampling point of Gauss-integration
	Me=Me + w*( rho*A*( (Nu'*Nu) + (Nv'*Nv) + (Nw'*Nw) ) + rho*I*( (Nvx'*Nvx)+(Nwx'*Nwx)+2*(Nphi'*Nphi) ) )*le/2;
	
	Be=Be+w*( 2*rho*A*Omega*( (Nv'*Nw) - (Nw'*Nv)))*le/2;
	
	Qe=Qe+w*(-rho*A*e*Omega^2*Nv')*le/2;
	%%
	Kteux=Kteux+w*A*E*((1+3*Ux+1.5*Ux^2+0.5*Vx^2+0.5*Wx^2)*Nux+(Vx+Vx*Ux)*Nvx+(Wx+Wx*Ux)*Nwx)'*Nux;
	Ktev=Ktev+w*rho*A*Omega^2*(-Nv'*Nv);
	Ktevx=Ktevx+w*A*E*((Vx+Vx*Ux)*Nux+(Ux+0.5*Ux^2+1.5*Vx^2+0.5*Wx^2)*Nvx+Vx*Wx*Nwx)'*Nvx;
	Ktevxx=Ktevxx+w*E*I*(Nvxx'*Nvxx);
	Ktew=Ktew+w*rho*A*Omega^2*(-Nw'*Nw);
	Ktewx=Ktewx+w*A*E*((Wx+Wx*Ux)*Nux+Vx*Wx*Nvx+(Ux+0.5*Ux^2+0.5*Vx^2+1.5*Wx^2)*Nwx)'*Nwx;
	Ktewxx=Ktewxx+w*E*I*(Nwxx'*Nwxx);
	Ktephi=Ktephi+w*rho*I*2*(-Nphix'*Nphix)*Omega^2;
	Ktephix=Ktephix+w*(G*It*Nphix)'*Nphix;
	
	end
	
	Kte=(Kteux+Ktev+Ktevx+Ktevxx+Ktew+Ktewx+Ktewxx+Ktephi+Ktephix)*le/2;
	Qe=diag(Qe);
	
	%% Ce, MeM, KteM
	xi=1;
	N=[ 0.5-xi/2, 0, 0, 0, 0, 0, 0.5+xi/2, 0, 0, 0, 0, 0; ...
	0, 1/2-(3*xi)/4+(xi^3)/4, 1/4-xi/4-(xi^2)/4+(xi^3)/4, 0, 0, 0, 0, 1/2+(3*xi)/4-(xi^3)/4, -1/4-xi/4+(xi^2)/4+(xi^3)/4, 0, 0, 0; ...
	0, -3/4+(3*xi^2)/4, -1/4-xi/2+(3*xi^2)/4, 0, 0, 0, 0, 3/4-(3*xi^2)/4, -1/4+xi/2+(3*xi^2)/4, 0, 0, 0; ...
	0, 0, 0, 1/2-(3*xi)/4+(xi^3)/4, 1/4-xi/4-(xi^2)/4+(xi^3)/4, 0, 0, 0, 0, 1/2+(3*xi)/4-(xi^3)/4, -1/4-xi/4+(xi^2)/4+(xi^3)/4, 0; ...
	0, 0, 0, -3/4+(3*xi^2)/4, -1/4-xi/2+(3*xi^2)/4, 0, 0, 0, 0, 3/4-(3*xi^2)/4, -1/4+xi/2+(3*xi^2)/4, 0; ...
	0, 0, 0, 0, 0, 0.5-xi/2, 0, 0, 0, 0, 0, 0.5+xi/2];
	
	Nu=N(1,:);
	Nw=N(2,:);
	Nv=N(4,:);
	
	MeM=m*( (Nu'*Nu) + (Nv'*Nv) + (Nw'*Nw) );
	
	BeM=2*m*Omega*( (Nv'*Nw) - (Nw'*Nv));
	
	QeM=-m*e*Omega^2*Nv';
	QeM=diag(QeM);
	
	Ce=c*(Nv'*Nv);
	
	KtevM=m*Omega^2*(-Nv'*Nv);
	KtewM=m*Omega^2*(-Nw'*Nw);
	
	KteM=KtevM+KtewM;
	
	end
	\end{lstlisting}
	